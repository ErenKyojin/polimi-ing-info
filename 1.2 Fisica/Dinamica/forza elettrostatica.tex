% Options for packages loaded elsewhere
\PassOptionsToPackage{unicode}{hyperref}
\PassOptionsToPackage{hyphens}{url}
%
\documentclass[
]{article}
\usepackage{amsmath,amssymb}
\usepackage{iftex}
\ifPDFTeX
  \usepackage[T1]{fontenc}
  \usepackage[utf8]{inputenc}
  \usepackage{textcomp} % provide euro and other symbols
\else % if luatex or xetex
  \usepackage{unicode-math} % this also loads fontspec
  \defaultfontfeatures{Scale=MatchLowercase}
  \defaultfontfeatures[\rmfamily]{Ligatures=TeX,Scale=1}
\fi
\usepackage{lmodern}
\ifPDFTeX\else
  % xetex/luatex font selection
\fi
% Use upquote if available, for straight quotes in verbatim environments
\IfFileExists{upquote.sty}{\usepackage{upquote}}{}
\IfFileExists{microtype.sty}{% use microtype if available
  \usepackage[]{microtype}
  \UseMicrotypeSet[protrusion]{basicmath} % disable protrusion for tt fonts
}{}
\makeatletter
\@ifundefined{KOMAClassName}{% if non-KOMA class
  \IfFileExists{parskip.sty}{%
    \usepackage{parskip}
  }{% else
    \setlength{\parindent}{0pt}
    \setlength{\parskip}{6pt plus 2pt minus 1pt}}
}{% if KOMA class
  \KOMAoptions{parskip=half}}
\makeatother
\usepackage{xcolor}
\setlength{\emergencystretch}{3em} % prevent overfull lines
\providecommand{\tightlist}{%
  \setlength{\itemsep}{0pt}\setlength{\parskip}{0pt}}
\setcounter{secnumdepth}{-\maxdimen} % remove section numbering
\ifLuaTeX
  \usepackage{selnolig}  % disable illegal ligatures
\fi
\IfFileExists{bookmark.sty}{\usepackage{bookmark}}{\usepackage{hyperref}}
\IfFileExists{xurl.sty}{\usepackage{xurl}}{} % add URL line breaks if available
\urlstyle{same}
\hypersetup{
  pdftitle={forza elettrostatica},
  hidelinks,
  pdfcreator={LaTeX via pandoc}}

\title{forza elettrostatica}
\author{}
\date{}

\begin{document}
\maketitle

Detta anche forza di Coulomb, due cariche elettriche puntiformi {} e {}
nel vuoto interagiscono con una forza diretta lungo la congiungente,
come descritto dalla legge di Coulomb:

legge di Coulomb

Date due carica {} e {}, ferme rispetto ad un osservatore in un sistema
inerziale nel vuoto, l\textquotesingle interazione elettrostatica tra {}
e {} è proporzionale al quadrato dalle loro cariche e diretta come la
congiungente tra le cariche stesse. Il verso dipende dal segno delle due
cariche.

\begin{itemize}
\tightlist
\item
  {} distanza tra le due cariche puntiformi
\item
  {} costante dielettrica del vuoto
\end{itemize}

La forza elettrostatica a differenza di quella gravitazionale può essere
attrattiva o repulsiva a seconda che le due cariche abbiano segni
discordi o concordi, il suo Modulo, come per la forza gravitazionale, è
inversamente proporzionale al quadrato della distanza

\end{document}
